\documentclass[sigconf]{acmart}

\usepackage{graphicx}
\usepackage{hyperref}
\usepackage{todonotes}

\usepackage{endfloat}
\renewcommand{\efloatseparator}{\mbox{}} % no new page between figures

\usepackage{booktabs} % For formal tables

\settopmatter{printacmref=false} % Removes citation information below abstract
\renewcommand\footnotetextcopyrightpermission[1]{} % removes footnote with conference information in first column
\pagestyle{plain} % removes running headers

\newcommand{\TODO}[1]{\todo[inline]{#1}}

\begin{document}
\title{Big Data in Media(Content) Industry}


\author{Sarang Fadnavis}
\orcid{HID-314}
\affiliation{%
  \institution{INDIANA UNIVERSITY BLOOMINGTON}
  \streetaddress{P.O. Box 1212}
  \city{Bloomington} 
  \state{INDIANA} 
  \postcode{47405}
}
\email{sfadnavi@iu.edu}

\begin{abstract}
Media industry has evolved from the traditional business model which was Television, Print Media (like Newspaper, articles, magazine) providing most of the revenues from it to new world of digital media. Digital media is the new era of content sharing on social networking or Internet. The rate at which internet is accessible to everyone as per IDC estimates, till 2020 – Business transaction will reach 450 Billion per day which reflect true business need and strategies can be planned as per market analysis 
\end{abstract}

\keywords{TRP = Television Rating Points, }

\maketitle

\section{Introduction}

News, TV, market analysis, Movies or Video on demand, etc were part of content distribution by specific organization / Production house to sell it to broadcaster to further use the content. All these distribution was one way communication and hardly possible to get exact Television Rating point (TRP) or feedback about article. Internet has given us flexibility to get feedback on the good / bad videos, Photos to reach 50million user Radio took 38 years, TV almost 13 years, Internet just 3years and facebook added 200 million user in less than a year. 
Such a heavy internet usage has raised a concern over data (content) availability and analysis to make it effective for marketing and strategic business as most of the data available is unstructured form. 
size \cite{Teaches}.

\section{Issues faced by Media industry}

\begin{itemize}
\item Audience Measurement methodologies and standardization of individual and multimedia channel planning
\item Tracking ROI and attribution across media channels ¬ proving effectiveness of media campaign activity
\item Keeping up to date with constantly evolving market trends, initiatives and opportunities
\item Business costs of continually evolving marketing technology and media research
\item Managing an increasingly diverse and Sliped agency roster
\item Fragmenting market and increased competition from other media and channels ie social, owned and earned media
\item Procurement and agency margins  finding an effective and sustainable business model
\item Disconnect and Mis-communication between client, agency and media
\item Limited scope for outside the square and new opportunities
\end{itemize}
size \cite{Failures}

\section{Big Data in Media Industry}

Internet has changed the world in terms of interaction and communication we do. Media contents are available on youtube, Netflix, chromecast for free or low cost .Also on social networking sites like facebook, twitter, Instagram and WhatsApp.
16 zettabytes unique data was created in 2106. Such an enormous amount of data can be analyze to get some answers to the change in pattern and product marketing

“We do not have a choice on whether we do social media the question is how well we do it”… Erik Qualman
                    
50 percent of world’s population is under 30 years and 96 percent of millennial are on social media, Facebook and Google tops highest weekly traffic on internet
size \cite{Stats}

\subsection{Benefits of big data and Internet}

Understand customer content preferences
\begin{itemize}
\item Increase the relevance in FT’s communication to customers
\item Personalize content
\item Deploy intelligence to customer touchpoints, including customer service, website, mobile apps and third parties, such as advertisers, in order to target campaigns
\end{itemize}

\subsection{statistics from 2017}

\begin{itemize} 
\item  3.8 Billion people uses internet
\item  20.8 Billion devices will be using internet by 2020
\item 24 hrs of video is uploaded to youtube every minute
\item 300 million photos uploaded to facebook everyday
\item 2 Billion active users in facebook
\end{itemize}
size \cite{Stats}

\section{Social Media analytics platform}

Comprehensive social media platform that combines data feed, data mining and data analysis tool
size \cite{SocialMedia}

\subsection{News Platform}

News analytics uses Natural language processing, technique to score news items and to confirm Author sentiments, Relevance, Volume Analysis, Uniqueness, and Headline Analysis

\subsection{Social Network media Platform}

Bandwidth, salesforce Marketing Cloud and Media Analysis Platform 
\begin{itemize}
\item Item type—stage of the story: Alert, Article, Updates or Corrections.
\item Item genre—classification of the story, i.e., interview, exclusive and wrap-up.
\item Headline—alert or headline text.
\item Relevance—varies from 0 to 1.0.
\item Prevailing sentiment—can be 1, 0 or −1.
\item Positive, neutral, negative—more detailed sentiment indication.
\item Broker action—denotes broker actions: upgrade, downgrade, maintain, and undefined or whether it is the broker itself
\end{itemize}

\section{Conclusion}

Content Media has exploded in terms of data generated with both structured and unstructured.
There is a need to use Big data and this need will be increasing day by day in order to work towards marketing and strategic planning new product companies will need the analytics for improving their product by understanding market and current trends.

\begin{acks}

  I would like to thank Prof  Gregor von Laszewski for the opportunity and Mani Kumar to help me with Git hub tool

\end{acks}

\bibliographystyle{ACM-Reference-Format}
\bibliography{report} 

\end{document}
